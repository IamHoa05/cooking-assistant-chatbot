\chapter{Giới thiệu}
Thông qua áp dụng các kiến thức thuộc lĩnh vực Trí tuệ nhân tạo (đặc biệt trong mảng Xử lý ngôn ngữ tự nhiên), trợ lý ảo thông minh CookGPT ra đời với mục đích hỗ trợ người dùng Việt Nam trở thành "master chef" một cách nhanh chóng, tiện lợi và thú vị.

Dự án Cooking Assistant Chatbot tập trung vào các lĩnh vực sau:

\begin{itemize}
    \item \textbf{Ngôn ngữ:} Sử dụng tiếng Việt, đảm bảo các hướng dẫn và gợi ý về món ăn phù hợp với người dùng địa phương.
    
    \item \textbf{Chức năng:} Hệ thống cung cấp gợi ý về món ăn, mô tả món ăn, hướng dẫn nấu ăn chi tiết, phân loại món theo độ khó, thời gian chế biến và khẩu phần. Hiện tại, hệ thống chưa hỗ trợ tạo món ăn mới hoặc fine-tune mô hình trực tiếp.
    
   
    \item \textbf{Ưu điểm của dự án:}
        \begin{itemize}
            \item Tích hợp các chức năng phong phú: phân loại món ăn theo độ khó, thời gian, khẩu phần, gợi ý món ăn từ nguyên liệu có sẵn.
            \item Giúp người dùng tiết kiệm thời gian tra cứu và tổng hợp từ nhiều nguồn: đưa ra danh sách kết quả phù hợp yêu cầu của người dùng với độ chính xác cao trong thời gian ngắn.
            \item Cung cấp các trải nghiệm tương tác vui vẻ, sinh động cho người dùng qua việc dùng giọng văn đậm chất Gen Z: thoải mái, gần gũi, hài hước và dễ hiểu.
        \end{itemize}
\end{itemize}


Với các mục tiêu và phạm vi trên, CookGPT hướng tới việc trở thành một trợ lý ẩm thực tiện ích, thân thiện, dễ sử dụng và luôn mang lại trải nghiệm ấn tượng cho người dùng trong suốt quá trình nghiên cứu, chế biến và thử nghiệm món ăn.
