\chapter*{Lời mở đầu}
Trong kỷ nguyên chuyển đổi số hiện nay, trí tuệ nhân tạo (Artificial Intelligence – AI) và các kỹ thuật xử lý ngôn ngữ tự nhiên (Natural Language Processing – NLP) đã trở thành nền tảng quan trọng trong việc phát triển các hệ thống tương tác người – máy. Bên cạnh các lĩnh vực phổ biến như chăm sóc khách hàng, giáo dục hay y tế, việc ứng dụng AI vào hỗ trợ nấu ăn đang thu hút sự quan tâm nhờ khả năng đáp ứng nhu cầu thực tiễn, mang tính cá nhân hóa và hỗ trợ tức thời cho người dùng.

\vspace{0.25cm}
Xuất phát từ nhu cầu tối ưu hóa trải nghiệm nấu ăn, đặc biệt trong bối cảnh người dùng ngày càng ưu tiên sự nhanh chóng và tiện lợi, nhóm thực hiện đề tài “Cooking Assistant Chatbot”. Hệ thống được xây dựng nhằm hỗ trợ người dùng tra cứu công thức nấu ăn, gợi ý món ăn dựa trên nguyên liệu sẵn có, hướng dẫn thao tác chế biến. Lựa chọn ứng dụng mô hình ngôn ngữ lớn (Large Language Models – LLMs), kết hợp thuật toán tìm kiếm, phân loại ý định (intent classification) và trích xuất thông tin, giúp chatbot có khả năng tương tác tự nhiên và đưa ra kết quả phù hợp với yêu cầu từ người dùng.

\vspace{0.25cm}
Báo cáo này trình bày toàn bộ quy trình triển khai hệ thống, bao gồm thiết kế kiến trúc, xây dựng tập dữ liệu, lựa chọn mô hình, phát triển thuật toán đề xuất món ăn và tích hợp backend – frontend. Đồng thời, báo cáo cũng đánh giá hiệu quả hệ thống thông qua các tiêu chí về độ chính xác, khả năng phản hồi và mức độ phù hợp với nhu cầu người dùng. Những kết quả đạt được không chỉ minh chứng cho tiềm năng ứng dụng AI trong lĩnh vực ẩm thực thế giới nói chung và ẩm thực Việt Nam nói riêng, mà còn mở ra nhiều hướng phát triển trong tương lai, đặc biệt trong việc nâng cao tính cá nhân hóa và tự động hóa quy trình nấu ăn trong cuộc sống hằng ngày.

\vspace{0.5cm}
Do hạn chế về kiến thức và thời gian, bài báo cáo này có thể còn nhiều thiếu sót. Chúng em rất mong nhận được sự góp ý và phản hồi để hoàn thiện hơn trong những lần báo cáo sau.

\vspace{0.5cm}
Chúng em xin chân thành cảm ơn!

\begin{flushright}
Hà Nội, ngày 30 tháng 11 năm 2025
\end{flushright}