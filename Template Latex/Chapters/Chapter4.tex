\chapter{Kết luận}

Hệ thống CookGPT đã chứng minh hiệu quả trong việc hỗ trợ người dùng tra cứu, gợi ý và cá nhân hóa trải nghiệm nấu ăn nhờ việc kết hợp đồng bộ các kỹ thuật tiền xử lý dữ liệu, embedding, xử lý ngôn ngữ tự nhiên và mô hình ngôn ngữ lớn. Quy trình tiền xử lý dữ liệu giúp chuẩn hóa và làm sạch tập dữ liệu công thức nấu ăn, đảm bảo thông tin được tổ chức nhất quán, có cấu trúc và giàu ngữ nghĩa. Bên cạnh đó, việc ứng dụng kỹ thuật embedding cho phép chatbot hiểu sâu hơn về mối quan hệ giữa nguyên liệu, phương pháp chế biến và khẩu vị, từ đó thực hiện tìm kiếm ngữ nghĩa chính xác và gợi ý món ăn phù hợp với yêu cầu của người dùng.

Các kỹ thuật xử lý ngôn ngữ tự nhiên kết hợp với mô hình ngôn ngữ lớn giúp chatbot có khả năng diễn giải và phản hồi linh hoạt, tự nhiên theo phong cách hội thoại của con người nói chung và gen Z nói riêng. Nhờ đó, chatbot không chỉ trả lời câu hỏi một cách linh hoạt, dí dỏm mà còn có thể hỗ trợ đưa ra thực đơn, danh sách nguyên liệu cần thiết, tối ưu hóa thời gian nấu và cá nhân hóa theo nhu cầu của từng người dùng thay vì chỉ là những câu chữ khô khan truyền thống.

Tổng thể, hệ thống thể hiện tính khả thi, hiệu quả và tính mở rộng cao. Trong tương lai, mô hình có thể được nâng cấp bằng cách tích hợp thêm dữ liệu hình ảnh món ăn ở từng giai đoạn nấu nhằm giúp người dùng có thể cải thiện chất lượng món ăn, cải thiện khả năng nhận diện nguyên liệu, tối ưu phương pháp truy xuất thông tin và bổ sung chức năng đề xuất thực đơn dựa trên chế độ ăn hoặc tình trạng sức khỏe. Ngoài ra, mong muốn phát triển mô hình hỗ trợ đa ngôn ngữ và giọng nói, phát triển thành ứng dụng di động để tăng khả năng tiếp cận với người dùng. Việc phát triển theo hướng đa phương thức và học tăng cường sẽ giúp CookGPT trở thành một trợ lý nấu ăn thông minh, toàn diện và thân thiện hơn với người dùng.

