\begin{frame}{Giới thiệu chung về embedding}
    \begin{itemize}
    \item Embedding là một kỹ thuật trong AI, học máy và xử lý ngôn ngữ tự nhiên dùng để biểu diễn dữ liệu dạng vector trong không gian nhiều chiều. 
    \item Embedding thể hiện mối quan hệ và mức độ tương đồng giữa các đối tượng thông qua khoảng cách hoặc góc trong không gian vector.
    \item Hỗ trợ máy tính nhận diện, phân loại, tìm kiếm thông tin hoặc dự đoán hiệu quả.
    \end{itemize}
\end{frame}

\begin{frame}{Nguyên lý hoạt động của embedding}
    \begin{itemize}
    \item Thông qua một mô hình học sâu được huấn luyện, dữ liệu được chuyển thành vector số trong một không gian liên tục.
    \end{itemize}
    \begin{center}
        \includegraphics[width=0.9\textwidth]{Slides/imgs/embedding.png}
        \\[4pt]
    \end{center}
\end{frame}

\begin{frame}{Các loại embedding phổ biến}
    \begin{itemize}
    \item \textbf{Word embedding}: biểu diễn các từ thành vector số: Word2Vec, GloVe. Nhận diện từ đồng nghĩa, phân loại từ theo ngữ cảnh. 
    \item \textbf{Sentence embedding} và \textbf{document embedding}: biểu diễn câu, tài liệu thành vector. Cho phép các mô hình hiểu nghĩa của cả câu, đoạn văn hoặc tài liệu dài: Sentence-BERT, BGE-M3.
    \item \textbf{Image embedding}: chuyển đổi hình ảnh thành vector số. Giúp máy tính nhận diện hình ảnh, tìm kiếm hình ảnh tương tự, phân loại hình ảnh theo nội dung.
    \item \textbf{Audio embedding}: biểu diễn âm thanh, giọng nói, nhạc thành vector số, hỗ trợ các ứng dụng nhận diện giọng nói, phân loại nhạc, phát hiện sự kiện âm thanh.
    \end{itemize}
\end{frame}

\begin{frame}{Mô hình embedding BGE-M3}
    \begin{itemize}
    \item Là mô hình sentence embedding đa chức năng được xây dựng trên kiến trúc Transformer, đặc điểm: 
    \begin{itemize}
        \item Đa ngôn ngữ: Hỗ trợ hơn 100 ngôn ngữ và ross-lingual.
        \item Đa chức năng: thực hiện đồng thời 3 loại truy xuất: tìm kiếm vector dày đặc, tìm kiếm vector thưa, tìm kiếm đa vector.
        \item Xử lý văn bản có độ dài khác nhau lên đến 8192 token.
        \item Huấn luyện dựa trên phương pháp self-knowledge distillation. 
         \end{itemize}
    \end{itemize}
    => Đáp ứng các yêu cầu của project: hỗ trợ tiếng Việt , cho kết quả tương đồng ngữ nghĩa chính xác, embedding các câu ngắn: tên món ăn, tên nguyên liệu.
\end{frame}

\begin{frame}{Quy trình xử lý văn bản trong BGE-M3}
    BGE-M3 sử dụng kiến trúc Transformer. Cụ thể thực hiện các bước:
    \begin{itemize}
        \item \textbf{Tokenization:} chia văn bản thành các token và mã hóa thành vector.
        \item \textbf{Token Embedding:} mỗi token được biểu diễn dưới dạng vector số.
        \item \textbf{Self-Attention:} xem xét mối quan hệ và mức độ phụ thuộc giữa các token trong câu.
        \item \textbf{Transformer Encoder:} biến đổi các vector nhúng thành biểu diễn ngữ cảnh của toàn câu.
    \end{itemize}
\end{frame}

\begin{frame}{Triển }
\begin{itemize}
    \item Đọc dữ liệu từ \texttt{recipes\_cleaned.csv}: \texttt{dish\_name}, \texttt{ingredient\_names}
          
    \item Load mô hình BGE-M3 từ \texttt{config.yml}
    \item Encode tên món ăn thành một vector 1024 chiều
    \item Encode nguyên liệu từng item riêng biệt thành mảng vector 1024 chiều
    \item Lưu kết quả vào file \texttt{recipes\_embeddings.pkl}
\end{itemize}
\end{frame}