\subsection*{Xây dựng hệ thống tìm kiếm với FAISS}

% Slide 1: Tổng quan FAISS
\begin{frame}{FAISS – Facebook AI Similarity Search}
\begin{itemize}
    \item Thư viện mã nguồn mở cho \textbf{vector similarity search}.
    \item Hỗ trợ \textbf{Exact / Approximate Nearest Neighbor}, GPU acceleration.
    \item Ứng dụng: \textbf{Semantic Search}, \textbf{Recommendation}, \textbf{RAG chatbot}.
\end{itemize}
\end{frame}

% Slide 2: Nhu cầu sử dụng
\begin{frame}{Nhu cầu sử dụng FAISS}
\begin{itemize}
\item \textbf{Vấn đề với từ khóa truyền thống:}
\begin{itemize}
    \item BM25, TF-IDF không hiểu \textbf{ngữ nghĩa}.
    \item Khó tìm món ăn tương tự khi query khác từ.
\end{itemize}

\vspace{0.5cm} % khoảng cách giữa 2 phần

\item \textbf{Giải pháp: Embedding + FAISS}
\begin{itemize}
    \item Chuyển text sang vector dense.
    \item Tìm nearest neighbors nhanh.
\end{itemize}

\end{itemize}
\end{frame}


% Slide 3: Kiến trúc FAISS
\begin{frame}{Kiến trúc FAISS}
\begin{itemize}
    \item \textbf{Index}: lưu trữ và truy xuất vector
    \item \textbf{Metric}: L2 distance / Inner Product
    \item \textbf{Quantizer}: cho approximate search
    \item \textbf{IVF}: chia không gian vector
    \item \textbf{GPU module}: tăng tốc dataset lớn
\end{itemize}
\end{frame}

% Slide 4: Các loại Index
\begin{frame}{Các loại Index trong FAISS}
\centering

\begin{block}{Exact Search}
\begin{itemize}
    \item IndexFlatIP, IndexFlatL2
    \item Ưu điểm: Độ chính xác 100\%
    \item Nhược điểm: Tốn CPU, dataset lớn
\end{itemize}
\end{block}

\begin{block}{Approximate Search}
\begin{itemize}
    \item IndexIVFFlat, IndexIVFPQ, HNSW
    \item Ưu điểm: Nhanh, tiết kiệm bộ nhớ
    \item Nhược điểm: Recall thấp hơn exact
\end{itemize}
\end{block}

\begin{block}{GPU Index}
\begin{itemize}
    \item GPUFlat, GPUIVF
    \item Ưu điểm: Tăng tốc truy vấn >< Nhược điểm: Cần GPU
\end{itemize}
\end{block}

\end{frame}

% Slide 5: Cơ chế hoạt động
\begin{frame}{Cơ chế hoạt động}
\centering
\begin{tikzpicture}[node distance=0.8cm, scale=0.8, every node/.style={scale=0.8}]
    \node[draw, rounded corners, fill=rosePink!20, text width=4cm, align=center] (input) {Input};
    \node[draw, rounded corners, fill=rosePink!20, below=of input, text width=4cm, align=center] (embed) {Embedding};
    \node[draw, rounded corners, fill=rosePink!20, below=of embed, text width=4cm, align=center] (index) {FAISS Index};
    \node[draw, rounded corners, fill=rosePink!20, below=of index, text width=4cm, align=center] (topk) {Top-k};
    \node[draw, rounded corners, fill=rosePink!20, below=of topk, text width=4cm, align=center] (rank) {Ranking};
    \node[draw, rounded corners, fill=rosePink!20, below=of rank, text width=4cm, align=center] (output) {Output};
    
    \draw[->, thick] (input) -- (embed);
    \draw[->, thick] (embed) -- (index);
    \draw[->, thick] (index) -- (topk);
    \draw[->, thick] (topk) -- (rank);
    \draw[->, thick] (rank) -- (output);
\end{tikzpicture}
\end{frame}




% Slide 6: Triển khai hệ thống
\begin{frame}{Triển khai FAISS trong hệ thống}
\begin{itemize}
    \item Chọn \textbf{IndexFlatIP} cho dataset vừa phải.
    \item Thêm vector embedding của ingredients và dish names.
    \item Top-k retrieval: \(\text{top\_k} = \underset{i \in [1,N]}{\mathrm{arg\,max}} (\mathbf{q} \cdot \mathbf{v}_i)\)
    \item Mapping top-k ID → dataset gốc.
\end{itemize}
\end{frame}

% Slide 7: Ưu điểm & Hạn chế
\begin{frame}{Ưu điểm \& Hạn chế của FAISS}
\begin{columns}[t] % căn trên để tiêu đề cùng hàng
    \column{0.48\textwidth}
    \centering
    \textbf{Ưu điểm}
    
    \vspace{0.3cm}
    \begin{itemize}
        \item Tốc độ cao, hỗ trợ Exact/Approx
        \item Hỗ trợ GPU
        \item Dễ tích hợp với text/hình ảnh/audio
        \item Batch search tối ưu throughput
    \end{itemize}

    \column{0.48\textwidth}
    \centering
    \textbf{Hạn chế}
    
    \vspace{0.3cm}
    \begin{itemize}
        \item Không tối ưu với dữ liệu sparse
        \item Approximate index recall thấp hơn exact
        \item Index lớn tốn RAM nếu không dùng IVF/PQ/GPU
    \end{itemize}
\end{columns}
\end{frame}


% Slide 8: Ứng dụng
\begin{frame}{Ứng dụng thực tế của FAISS}
\begin{itemize}
    \item Chatbot / Search Engine (RAG)
    \item Recommendation System
    \item Image Similarity (CLIP/CNN)
    \item Semantic Search NLP
    \item Multimodal Search (text, hình ảnh, audio)
\end{itemize}
\end{frame}

% Slide: Hiệu suất tìm kiếm với FAISS
\begin{frame}{Hiệu suất tìm kiếm với FAISS}
\begin{itemize}
    \item \textbf{Index dish:} 2420 vectors, 1024 chiều
    \begin{itemize}
        \item Độ trễ trung bình: 1.10 ms/query
        \item Truy vấn mỗi giây(QPS): 910
        \item Precision@5 / Recall@5: 100\%
    \end{itemize}
    
    \item \textbf{Index ingredient names:} 15752 vectors, 1024 chiều
    \begin{itemize}
        \item Độ trễ trung bình: 3.85 ms/query
        \item Truy vấn mỗi giây(QPS): 260
        \item Precision@5 / Recall@5: 24\%
    \end{itemize}
\end{itemize}

\vspace{0.3cm}
\textbf{Nhận xét:}
\begin{itemize}
    \item Index dish rất nhanh và chính xác → phù hợp với truy vấn món ăn.
    \item Index ingredient names chậm hơn, độ chính xác thấp hơn → cần tối ưu hoặc kết hợp với các bộ lọc khác.
\end{itemize}
\end{frame}

