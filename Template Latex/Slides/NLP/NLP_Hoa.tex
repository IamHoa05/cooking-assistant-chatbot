\section*{Xác dịnh intent}

% Slide 1: Mục tiêu
\begin{frame}{Mục tiêu xác dịnh intent}
\textbf{Xác định intent} nhằm phân loại chính xác ý định của người dùng khi tương tác với hệ thống:
\begin{itemize}
    \item Nhận biết người dùng muốn \textbf{tìm kiếm món ăn}.
    \item Nhận biết người dùng muốn \textbf{hướng dẫn nấu ăn}.
\end{itemize}
Việc nhận diện intent chính xác tạo nền tảng cho các bước tiếp theo như trích xuất thông tin và gợi ý món ăn.
\end{frame}

% Slide 2: Pipeline
\begin{frame}{Pipeline xác định intent}
\begin{figure}
    \centering
    \includegraphics[width=12.5cm, height=3.2cm]{Slides/imgs/pipeline_intent.png}
    \caption{Pipeline xác định intent}
\end{figure}
\end{frame}

% Slide 3: Thu thập dữ liệu Intent
\begin{frame}{Thu thập và xây dựng tập dữ liệu Intent}
Hai nhóm intent chính:
\begin{itemize}
    \item \texttt{suggest\_dishes} (gợi ý món ăn)
    \item \texttt{cooking\_guide} (hướng dẫn nấu ăn)
\end{itemize}

Quy trình xây dựng dữ liệu:
\begin{enumerate}
    \item Mô tả intent ban đầu.
    \item Sinh câu mẫu bằng mô hình ngôn ngữ (đa dạng, paraphrase, dễ gây nhầm lẫn).
    \item Kiểm duyệt thủ công (loại bỏ lỗi, trùng lặp, nhiễu).
    \item Mở rộng dữ liệu trong quá trình kiểm thử.
\end{enumerate}

Kết quả: khoảng \textbf{1.000 mẫu intent} đa dạng.
\end{frame}

% Slide 4: Phương pháp Hybrid
\begin{frame}{Phương pháp Hybrid}
Hệ thống kết hợp \textbf{embedding ngữ nghĩa} và \textbf{rule-based từ khóa}:
\begin{enumerate}
    \item Biểu diễn ngữ nghĩa bằng embedding (BGE-M3).
    \item Rule-based từ từ khóa đặc trưng cho từng intent.
    \item Công thức kết hợp điểm:
    \[
    S = s_{\text{embed}} + \alpha \cdot s_{\text{rule}}, \quad \alpha = 0.1
    \]
\end{enumerate}
\end{frame}


% Slide 6: Triển khai
\begin{frame}{Cách triển khai}
Các bước triển khai:
\begin{enumerate}
    \item Tiếp nhận câu truy vấn.
    \item Sinh embedding bằng BGE-M3.
    \item Tính toán top-k mean cosine.
    \item Áp dụng rule-based từ khóa.
    \item Tổng hợp điểm và chọn intent.
\end{enumerate}
\end{frame}

% Slide 7: Ưu điểm
\begin{frame}{Ưu điểm}
\begin{itemize}
    \item Linh hoạt khi mở rộng thêm intent mới.
    \item Dễ điều chỉnh bằng cách thay đổi trọng số hoặc danh sách từ khóa.
    \item Độ chính xác cao với câu truy vấn tiếng Việt tự nhiên.
\end{itemize}
\end{frame}

% Slide 1: Mục tiêu
\begin{frame}{Mục tiêu Slot Extraction}
\begin{itemize}
    \item Trích xuất thông tin quan trọng từ câu nhập của người dùng.
    \item Các slot chính:
    \begin{itemize}
        \item \textbf{ingredient}: nguyên liệu (gà, khoai tây, cà rốt)
        \item \textbf{category}: danh mục món ăn (kho, xào, luộc)
        \item \textbf{dish\_name}: tên món cụ thể (bánh mì chiên bơ tỏi)
        \item \textbf{time}: thời gian nấu (45 phút, 2 giờ)
        \item \textbf{difficulty}: độ khó (dễ, trung bình, khó)
        \item \textbf{serving}: số khẩu phần (4 người, 2 phần)
    \end{itemize}
    \item Mục tiêu: nhận diện chính xác, tránh nhầm substring, để làm input phục vụ quá trình tìm kiếm sau
\end{itemize}
\end{frame}

% Slide 2: Tokenization
\begin{frame}{Phương pháp: Tokenization}
\begin{itemize}
    \item Chia nhỏ văn bản thành token:
    \begin{itemize}
        \item Word-level: ``Tôi muốn ăn cơm'' $\rightarrow$ [Tôi, muốn, ăn, cơm]
        \item Character-level: ``cơm'' $\rightarrow$ [c, ơ, m]
        \item Subword/BPE: ``cooking'' $\rightarrow$ [cook, ing]
        \item Sentence-level: ``Tôi thích NLP. Nó thú vị.'' $\rightarrow$ [Tôi thích NLP., Nó thú vị.]
    \end{itemize}
    \item Mục đích: chuẩn hóa dữ liệu, phục vụ embeddings/NLP.
\end{itemize}
\end{frame}

% Slide 3: Thư viện Tokenization
\begin{frame}{Thư viện Tokenization}
\centering
\begin{tabularx}{\textwidth}{|l|l|X|}
\hline
\textbf{Thư viện} & \textbf{Ngôn ngữ} & \textbf{Cách hoạt động} \\
\hline
spaCy & Python & NLP pipeline, nhận diện từ, số, tên riêng, chuẩn hóa văn bản \\
ViTokenizer & Python & Tokenize tiếng Việt theo từ, dictionary + rule-based \\
BERT/HuggingFace & Python & Subword tokenization (BPE/WordPiece), tạo token ID cho embeddings \\
\hline
\end{tabularx}
\end{frame}

% Slide 4: Lý do chọn ViTokenizer
\begin{frame}{Lý do chọn ViTokenizer}
\begin{itemize}
    \item \textbf{spaCy}: mạnh mẽ đa ngôn ngữ nhưng không tối ưu cho tiếng Việt.
    \item \textbf{ViTokenizer}: chuyên biệt cho tiếng Việt, xử lý từ ghép, phù hợp nhận diện slot.
    \item \textbf{BERT Tokenizers}: chuẩn cho embeddings, dùng kết hợp sau khi tokenize.
\end{itemize}
\end{frame}

% Slide 5: N-grams
\begin{frame}{Phương pháp: N-grams}
\begin{itemize}
    \item Tạo ra tập con liên tiếp của token (1 đến N).
    \item Ưu tiên n-grams dài để tránh nhầm lẫn.
    \item Kết hợp từ điển để phát hiện slot chính xác.
\end{itemize}
\end{frame}

% Slide 6: Fuzzy Matching
\begin{frame}{Phương pháp: Fuzzy Matching}
\begin{itemize}
    \item Là phương pháp so khớp gần đúng với từ điển, không yêu cầu trùng tuyệt đối.
    \item Xử lý biến thể ngôn ngữ, lỗi chính tả.
    \item Ví dụ: ``khoai tay'' $\rightarrow$ ``khoai tây''.
    \item Dựa trên similarity ratio, chỉ match khi vượt ngưỡng.
\end{itemize}
\end{frame}

% Slide 7: Rule-based Extraction
\begin{frame}{Phương pháp: Rule-based Extraction}
\begin{itemize}
    \item Dùng regex/từ khóa đặc trưng cho slot có cấu trúc chuẩn.
    \item Ví dụ:
    \begin{itemize}
        \item ``2 giờ 30 phút'' $\rightarrow$ 150 phút.
        \item ``4 người'' $\rightarrow$ số khẩu phần 4.
    \end{itemize}
    \item Kết hợp tokenization + n-grams để tăng độ chính xác.
\end{itemize}
\end{frame}

% Slide 8: Triển khai
\begin{frame}{Cách triển khai}
\begin{itemize}
    \item Tokenization bằng ViTokenizer.
    \item Sinh n-grams (1–6 token).
    \item So khớp nguyên liệu với từ điển + fuzzy matching.
    \item Regex cho category, thời gian, khẩu phần, độ khó.
    \item Gom slot theo intent:
    \begin{itemize}
        \item \texttt{suggest\_dishes}: category, ingredient, time, difficulty, serving.
        \item \texttt{cooking\_guide}: dish\_name.
    \end{itemize}
\end{itemize}
\end{frame}

% Slide 9: Ưu điểm và Nhược điểm
\begin{frame}{Ưu điểm và Nhược điểm}
\textbf{Ưu điểm:}
\begin{itemize}
    \item Trích xuất chính xác slot quan trọng.
    \item Kết hợp token-level + regex giảm nhầm substring.
    \item Fuzzy matching nhận diện tên món gần đúng.
\end{itemize}

\textbf{Nhược điểm:}
\begin{itemize}
    \item Category/nguyên liệu dựa vào từ điển → không nhận diện từ mới.
    \item Cần kết hợp embedding + cosine similarity để xử lý tự nhiên hơn.
    \item Cần rule ranking hoặc confidence score để ưu tiên slot đáng tin cậy.
\end{itemize}
\end{frame}



